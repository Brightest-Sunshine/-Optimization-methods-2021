\documentclass{article}
\usepackage[utf8]{inputenc}

\documentclass[a4paper]{article}
\usepackage[12pt]{extsizes}
\usepackage{amsthm, amssymb, amsmath, amsfonts, nccmath, empheq}
\usepackage{float}
\usepackage[hidelinks]{hyperref} 
\usepackage{color,colortbl} 
\renewcommand{\labelenumii}{\Roman{enumii}}
\usepackage[warn]{mathtext}
\usepackage[T1,T2A]{fontenc}
\usepackage[utf8]{inputenc}
\usepackage[english,russian]{babel}
\usepackage{tocloft}
\linespread{1.5}
\usepackage{indentfirst}
\usepackage{setspace}
%\полуторный интервал
\onehalfspacing


\newtheorem{theorem}{Теорема}
\newtheorem{definition}{Опредление}
\newtheorem{corollary}{Следствие}[theorem]
\newtheorem{lemma}{Лемма}


\newcommand{\RomanNumeralCaps}[1]
    {\MakeUppercase{\romannumeral #1}}

\usepackage{amssymb}

\usepackage{graphicx, float}
\graphicspath{{pictures/}}
\DeclareGraphicsExtensions{.pdf,.png,.jpg}
\usepackage[left=20mm,right=1cm,
    top=2cm,bottom=20mm,bindingoffset=0cm]{geometry}
\renewcommand{\cftsecleader}{\cftdotfill{\cftdotsep}}

\addto\captionsrussian{\renewcommand{\contentsname}{СОДЕРЖАНИЕ}}

\usepackage{fancyhdr}
\usepackage[nottoc]{tocbibind}

\fancypagestyle{plain}{%
\fancyhf{}
\renewcommand{\headrulewidth}{0pt}
\fancyhead[R]{\thepage}
}

\usepackage{blindtext}
\pagestyle{myheadings}
% href
\usepackage{hyperref}
\setcounter{MaxMatrixCols}{20}





\begin{document}
\begin{titlepage}
  \begin{center}
  
     
    \large
    
    Санкт-Петербургский политехнический университет Петра Великого
    
    Институт прикладной математики и механики
    
    \textbf{Высшая школа прикладной математики и вычислительной физики}
    
    \vfill
     
     
    \textsc{\textbf{\Large{Лабораторная работа №3}}}\\[5mm]
     
    {\large \textbf{Тема: <<Решение задач одномерной минимизации>>}}
    
    \\ по дисциплине\\ <<Методы оптимизаций>>\\

\end{center}

\vfill


\begin{tabular}{l p{140} l}
Выполнили студенты \\группы 3630102/80401   

&  &Мамаева Анастасия Сергеевна\\
&  &Веденичев Дмитрий Александрович\\
&  &Тырыкин Ярослав Алексеевич\\

Руководитель\\Доцент, к.ф.-м.н.& \hspace{0pt} &   Родионова Елена Александровна \\\\
\end{tabular}

\hfill \break
\hfill \break
\begin{center} Санкт-Петербург \\2021 \end{center}
\thispagestyle{empty}
 
\end{titlepage}
\newpage
\begin{center}
    \setcounter{page}{2}
    \tableofcontents  
\end{center}

\newpage
\section{Постановка задачи}
\noindent Пусть имеются две трансцендентные функции: 
$$f(x)~=~x^{2}-2x+e^{-x}~~[1;~1.5]~~\varepsilon=0.05$$
$$f(x)~=~x^{3}-3sin(x)~~~[0.5;~1]~~~\varepsilon=0.05 $$

\noindent Необходимо:
\begin{enumerate}
    \item Найти минимум данных функций, используя методы дихотомии и Фибоначчи.
    \item Провести сравнительный анализ данных методов.
    \item Исследовать зависимость числа обращений к функции от задаваемой точности.
    \item Вывести аналитическую оценку  для числа обращений к функции у обоих методов.
\end{enumerate}

\section{Применимость методов}
\noindent Рассмотрим простейшую  математическую модель оптимизации, в которой целевая функция зависит от одной переменной, а допустимым множеством является отрезок вещественной оси:
$$f(x)\longrightarrow min$$
$$x \in [a;~b]$$
Для решения таких задач на практике, как правило, применяются приближённые методы. Они позволяют найти решение этой задачи с необходимой точностью в результате определения конечного числа значений функции $f(x)$ и её производных в некоторых точках отрезка $[a;~b]$. Методы использующие только значения функции и не требующие вычисления её производных называются прямыми методами.
\\\\
\noindent Большим достоинством прямых методов является то, что от целевой функции не требуется дифференцируемости и, более того, она может быть не задана в аналитическом виде. Единственное, на чём основаны алгоритмы прямых методов минимизации, это возможность определения значений $f(x)$ в заданныых точках.
\\\\
\noindent Рассмотрим метод дихотомии и Фибоначчи. Самым слабым требованием на функцию $f(x)$, позволяющим использовать эти методы, является, является её унимодальность.
\begin{definition}
Функция $f(x)$ называется унимодальной, если для $x\in [a;~b]$ существует единственная точка глобального минимума, слева от которой  $f(x)$ монотонно убывает, а справа - монотонно возрастает.
\end{definition}
\\\\
\noindent Поэтому проверим выполнение данного условия для наших функций. Для этого построим их графики.
\begin{figure}[H]
\center{\includegraphics[scale=0.80]{Figure_1.jpeg}}
\label{fig:image}
\end{figure}
\begin{figure}[H]
\center{\includegraphics[scale=0.80]{Figure_2.jpeg}}
\label{fig:image}
\end{figure}
\noindent Видим, что функции являются унимодальными, значит для них можем применить прямые методы поиска минимального значения.

\section{Описание алгоритмов}
\subsection{Метод дихотомии}
\noindent Пусть $a < x_1 < x_2 < b$. Сравнив значения $f(x)$ в точках $x_1$ и $x_2$, можно сократить отрезок поиска точки $x^{*}$, перейдя к отрезку $[a;~x_2]$, если $f(x_1)\le f(x_2)$, или отрезку $[x_1;~b]$, если $f(x_1)>f(x_2)$. Описанную процедуру можно повторить необходимое число раз, последовательно уменьшая отрезок, содержащий точку минимума. Когда длина последнего из найденных отрезков станет достаточно малой, следует положить $x^{*} = \bar x$, где $\bar x$ -- одна из точек этого отрезка, например его середина.
\\\\
\noindent В методе дихотомии точки $x_1~и~x_2$ располагаются близко к середине очередного отрезка $[a;~b]$, т.е
$$x_1=\frac{b+a-\delta}{2}$$
$$x_2=\frac{b+a+\delta}{2}$$
где $\delta>0$ -- малое число.
\\\\
\noindentВ конце вычислений в качестве приближённого значения $x^{*}$ берут середину последнего из найденных отрезков $[a;~b]$, убедившись предварительно, что достигнуто неравенство $b-a \le \varepsilon$. Опишем алгоритм метода
\begin{enumerate}
    \item Вычисляем $x_1=\frac{b+a-\delta}{2}$ и $x_2=\frac{b+a+\delta}{2}$. Вычисляем $f(x_1)$ и $f(x_2)$.
    \item Сравниваем $f(x_1)~и~ f(x_2)$. Если $f(x_1)\le f(x_2)$, то перейти к отрезку $[a;~x_2]$, положив $b=x_2$, иначе -- к отрезку $[x_1;~b]$, положив $a=x_1$.
    \item Найти достигнутую точность $\varepsilon_n=b-a$. Если $\varepsilon_n>\varepsilon$, то перейти к следующей итерации, вернувшись к шагу 1. Если $\varepsilon_n\le \varepsilon$, то завершить поиск $x^{*}$, перейдя к шагу 4.
    \item Положить $x^{*}=\bar x = \frac{a+b}{2},~~f^{*}=f(\bar x)$
\end{enumerate}
\subsection{Выбор числа $\delta$}
\noindent Число $\delta$ выбирается на интервале $(0;\varepsilon)$ с учётом следующих соображений:
\begin{enumerate}
    \item Чем меньше $\delta$, тем больше относительное уменьшение длины отрезка на каждой итерации, т.е. при уменьшении $\delta$ достигается более высокая скорость сходимости метода дихотомии.
    \item При чрезмерно малом $\delta$ сравнение значений $f(x)$ в точках $x_1~и~x_2$, отличающихся на величину $\delta$, становится затруднительным. Поэтому выбор $\delta$ должен быть согласован с точностью определения $f(x)$ и с количеством верных десятичных знаков при задании аргумента $x$.
    \item В нашей работе будем использовать $\delta$ равную 0.1\% от текущей длины отрезка $[a_k;~b_k]$
\end{enumerate}
\subsection{Метод Фибоначчи}
\noindent Суть заключается в том, что здесь генерируется последовательность чисел Фибоначчи:
$$F_0=F_1=1$$
$$F_k_+_1=F_k+F_k_-_1$$
\begin{enumerate}
		\item Выбираем допустимую конечную длину интервала неопределенности $l$ и константу различимости $\varepsilon$ 
		\item Выбираем общее число вычислений функции $n$ так, что $F_n >\frac{b-a}{l}$
		\item Положим $\lambda_1 = a_1+\frac{F_{n-2}}{F_n}(b-a)$ и  $\mu_1 = a_1+\frac{F_{n-1}}{F_n}(b-a)$
		\item Вычислим значения функции в точках $\lambda_1$ и $\mu_1$
		\item Положим $k = 1$
		\item Если значение функции в точке $\lambda_k$ больше, чем в точке $\mu_k$ перейдем к пункту 7, иначе к пункту 8
		\item Положим $a_{k+1} = \lambda_k$, $b_{k+1} = b_k$, $\lambda_{k+1} = \mu_k$, $\mu_{k+1} = a_{k+1}+\frac{F_{n-k-1}}{F_{n-k}}(b_{k+1}-a_{k+1})$\\
		Если $k = n - 2$, то перейдем к пункту 10, иначе вычисляем функцию в точке $\mu_{k+1}$ и переходим к пункту 9
		\item Положим $a_{k+1} = a_k$, $b_{k+1} = \mu_k$, $\mu_{k+1} = \lambda_k$, $\lambda_{k+1} = a_{k+1}+\frac{F_{n-k-2}}{F_{n-k}}(b_{k+1}-a_{k+1})$\\
		Если $k = n - 2$, то перейдем к пункту 10, иначе вычисляем функцию в точке $\mu_{k+1}$ и переходим к пункту 9
		\item Заменяем $k$ на $k+1$ и переходим к пункту 6
		\item Положим $\lambda_{n} = \lambda_{n-1}$, $\mu_{n} = \lambda_n + \varepsilon$. Если функция в точке $\lambda_n$ равна функции в точке $\mu_n$, то положим $a_n = \lambda_n$, $b_n=b_{n-1}$, иначе $a_n = a_{n-1}$, $b_n=\mu_n$. Повторяем пока заданный интервал неопределенности не удовлетворяет заданной точности.
	\end{enumerate}
	После того, как мы этот метод реализуем многократно, мы получим, что метод сходится в одну точку и она будет иметь следующие координаты $x^{*}=\frac{a_{n-1}+b_{n-1}}{2}$

\section{Результаты решения задачи}
\noindentДля функции $f(x)~=~x^{2}-2x+e^{-x}$ на интервале $[1;~1.5]$ при заданной точности $~\varepsilon=0.05$ метод дихотомии дал результат:
$$x^{*} = 1.172$$
$$f(x^{*}) = -0.661$$
метод Фибоначчи для этой же функции выдал значения, равные: 
$$x^{*} = 1.179$$
$$f(x^{*}) = -0.660$$
Если рассматривать функцию $f(x)~=~x^{3}-3sin(x)~~~[0.5;~1]~~~\varepsilon=0.05 $, то методом дихотомии получим результат:
$$x^{*} = 0.828$$
$$f(x^{*})=-1.642$$
методом Фибоначчи:
$$x^{*} = 0.845$$
$$f(x^{*})=-1.641$$

\section{Аналитическая оценка числа обращений к функции}

\subsection{Аналитическая оценка числа обращений к функции метода дихотомии}
\noindent Обозначим длину исходного отрезка $[a;~b]$ через $\Delta_0$. Длина отрезка, полученного после первой итерации, будет $$\Delta_1=\frac{\Delta_0}{2}+\frac{\delta}{2}$$ после второй итерации $$\Delta_1=\frac{\Delta_1}{2}+\frac{\delta}{2}=\frac{b-a}{4}+\delta(\frac{1}{4}+\frac{1}{2})$$ после третьей $$\Delta_3=\frac{\Delta_2}{2}+\frac{\delta}{2}=\frac{b-a}{8}+\delta(\frac{1}{8}+\frac{1}{4}+\frac{1}{2})$$ и т.д.
Таким образом, в результате $n$ итераций длина отрезка поиска точки $x^{*}$ станет $$\Delta_n=\frac{b-a}{2^n}+(\frac{1}{2^n}+\frac{1}{2^{n-1}}+\dots +\frac{1}{2})=\frac{b-a}{2^n}+(1-\frac{1}{2^n})\delta$$
При этом будет достигнута точность определения точки минимума $\varepsilon_n=\Delta_n$
$$\varepsilon_n=\frac{b-a}{2^n}+(1-\frac{1}{2^n})\delta \le \varepsilon$$
$$\frac{b-a}{2^n}+\delta-\frac{\delta}{2^n} \le \varepsilon$$
$$\frac{b-a}{2^n}-\frac{\delta}{2^n} \le \varepsilon -\delta$$
$$\frac{b-a-\delta}{2^n} \le \varepsilon -\delta$$
$$\frac{b-a-\delta}{\varepsilon -\delta} \le 2^n$$
$$n \ge log_2\frac{b-a-\delta}{\varepsilon-\delta}$$
Получили формулу для подсчёта числа итераций. 
Подставим в полученную формулу наши данные:
$$f(x)~=~x^{2}-2x+e^{-x}~~[1;~1.5]~~\varepsilon=0.05$$
$$n \ge log_2\frac{1.5-1-0.0005}{0.05-0.0005}=3.32$$
Значит $n=4$. Теперь учтём, что на каждой итерации обращаемся к функции ровно два раза, значит $m=2n$. Таким образом число вызовов функции равняется 8. \\\\
Повторим те же действия для второй функции: 
$$f(x)~=~x^{3}-3sin(x)~~~[0.5;~1]~~~\varepsilon=0.05 $$
$$n \ge log_2\frac{1-0.5-0.0005}{0.05-0.0005}=3.32$$
Аналогично предыдущему получаем $n = 4,~m = 8.$
\\\\
Результаты, полученные с помощью программного кода, полностью совпали с нашими ожиданиями.

\subsection{Аналитическая оценка числа обращений к функции метода Фибоначчи}

\noindent Число итераций, необходимое для достижения заданной точности $\varepsilon~$, можно найти из условия $\varepsilon_n \le \varepsilon$ с учётом соотношения 
$$\varepsilon_k = b_{k-1}-a_{k-1}=\frac{F_{n-k}}{F_{n-k+1}}\frac{F_{n-(k-1)}}{F_{n-(k-2)}}(b_{k-1}-a_{k-1}) = \frac{F_{n-k}}{F_n} (b-a)$$
Пусть $k = n - 1$, тогда:\\
$$\varepsilon \le \frac{b-a}{F_n}$$
Можем показать, что $$\frac{1}{F_n} \longrightarrow (\frac{\sqrt{5}-1}{2})^{n-1}=(0.61803)^{n-1}$$
Тогда имеем:
$$\varepsilon \le (b-a)(0.61803)^{n-1}$$
$$0.61803\cdot \varepsilon \le (b-a)(0.61803)^{n}$$
$$\frac{0.61803\cdot \varepsilon}{b-a} \le (0.61803)^n $$
$$log_2\frac{0.61803\cdot \varepsilon}{b-a} \le log_2(0.61803)^n $$
$$log_2\frac{0.61803\cdot \varepsilon}{b-a} \le n\cdot log_2(0.61803)$$
$$n \ge \frac{log_2\frac{0.61803\cdot \varepsilon}{b-a}}{log_2(0.61803)}$$
Подставим в полученную формулу наши значения:
$$f(x)~=~x^{2}-2x+e^{-x}~~[1;~1.5]~~\varepsilon=0.05$$
$$n \ge \frac{log_2(\frac{0.61803\cdot 0.05}{1.5-1})}{log_2(0.61803)} = \frac{log_2(0.061803)}{log_2(0.61803)}=5.78$$
Если число итераций $n$, то число вызова функции будет $m=n+1$, так как на первой итерации необходимо вычислить значение функции в 2-ух точках. Значит $n = 6$, а $m = 7$.\\
Аналогичные действия проделаем для второй функции: 
$$f(x)~=~x^{3}-3sin(x)~~~[0.5;~1]~~~\varepsilon=0.05 $$
$$n \ge \frac{log_2(\frac{0.61803\cdot 0.05}{1-0.5})}{log_2(0.61803)} = \frac{log_2(0.061803)}{log_2(0.61803)}=5.78$$
$n=6~и~m=7$\\
Фактическое и теоретическое количество вызовов функции на нашем примере совпало.

\section{Сравнительный анализ числа обращений от заданной точности}
\noindent Проведём ряд экспериментов, на основе которых составим таблицы фактического обращения к функции от заданного числа $\varepsilon$.
\begin{table}[H]
    \centering
    \begin{tabular}{|c|c|c|c|c|}
    \hline
         \varepsilon & результат $f(x^{*})$ & число обращений & коэффициент сжатия \\ \hline
            \multicolumn{4}{c}{метод дихотомии}\\ \hline
     0.1 & -0.66 & 6 &  0.125751\\ \hline
     0.01 & -0.661 & 12 & 0.01581\\ \hline
     0.001 & -0.6609 & 18 & 0.001988\\ \hline
      \multicolumn{4}{c}{метод Фибоначчи}\\ \hline
      0.1 & -0.66 & 5 &0.125\\ \hline
      0.01 & -0.661 & 10 &0.0112\\ \hline
      0.001 & -0.6609 & 15 &0.00101\\ \hline
    \end{tabular}
    \caption{$f(x)=x^2-2x+e^{-x}$}
    \label{tab: tab1}
\end{table}

\begin{table}[H]
    \centering
    \begin{tabular}{|c|c|c|c|c|}
    \hline
         \varepsilon & результат $f(x^{*})$ & число обращений & коэффициент сжатия \\ \hline
            \multicolumn{4}{c}{метод дихотомии}\\ \hline
     0.1 & -1.64 & 6 &0.125751\\ \hline
     0.01 & -1.642 & 12 &0.01581\\ \hline
     0.001 & -1.6421 & 18 & 0.001988\\ \hline
      \multicolumn{4}{c}{метод Фибоначчи}\\ \hline
      0.1 & -1.64 & 5 &0.125\\ \hline
      0.01 & -1.642 & 10 &0.0112\\ \hline
      0.001 & -1.6421 & 15 &0.00101\\ \hline
    \end{tabular}
    \caption{$f(x)=x^3-3sin(x)$}
    \label{tab: tab2}
\end{table}


\section{Оценка достоверности полученного результата}

\begin{lemma}
Пусть $f(x)$ -- унимодальная на $[a;~b]$. Пусть $x_1,~x_2 \in [a;~b]~и~x_1<x_2$. Тогда:
\begin{itemize}
    \item если $f(x_1)\ge f(x_2)$, то $x^{*}\notin[a;~x_1]$
    \item если $f(x_1)\le f(x_2)$, то $x^{*}\notin[x_2;~b]$
\end{itemize}
\end{lemma}
\noindent Воспользуемся пакетом MATLAB2020b и найдем минимум функции на заданном отрезке
$$f(x)~=~x^{2}-2x+e^{-x}~~[1;~1.5]$$
\begin{figure}[H]
\center{\includegraphics[scale=0.42]{MATLAB1.jpg}}
\label{fig:image}
\end{figure}
\noindent Теперь проверим, что найденное значение действительно принадлежит всем интервалам неопределённости, которые находит наша программа метода дихотомии:
\begin{center}
1.1572 \in [1, 1.5]\\
1.1572 \in [1, 1.2505]\\
1.1572 \in [1.1249995, 1.2505]\\
1.1572 \in [1.1249995, 1.1878752505]\\
1.1572 \in [1.1563744994995, 1.1878752505]\\    
\end{center}
Аналогично проверим для метода Фибоначчи:
\begin{center}
1.1572 \in [1, 1.5]\\
1.1572 \in [1, 1.3095238095238095]\\
1.1572 \in [1.119047619047619, 1.3095238095238095]\\
1.1572 \in [1.119047619047619, 1.2380952380952381]\\
1.1572 \in [1.119047619047619, 1.1904761904761905]\\
1.1572 \in [1.1428571428571428, 1.1904761904761905]\\
1.1572 \in [1.1666666666666667, 1.1904761904761905]\\
\end{center}
Для второй функции проведем аналогичные выкладки:
$$f(x)~=~x^{3}-3sin(x)~~~[0.5;~1] $$
\begin{figure}[H]
\center{\includegraphics[scale=0.42]{MATLAB2.jpg}}
\label{fig:image}
\end{figure}
\noindentМетод дихотомии:
\begin{center}
0.8257 \in [0.5, 1]\\
0.8257 \in [0.7495, 1]\\
0.8257 \in [0.7495, 0.8750005000000001]\\
0.8257 \in [0.8121247495, 0.8750005000000001]\\
0.8257 \in [0.8121247495, 0.8436255005005001]\\
\end{center}
Метод Фибоначчи:
\begin{center}
0.8257 \in [0.5, 1]\\
0.8257 \in [0.6904761904761905, 1]\\
0.8257 \in [0.6904761904761905, 0.8809523809523809]\\
0.8257 \in [0.7619047619047619, 0.8809523809523809]\\
0.8257 \in [0.8095238095238095, 0.8809523809523809]\\
0.8257 \in [0.8095238095238095, 0.8571428571428571]\\
0.8257 \in  [0.8333333333333333, 0.8571428571428571]\\
\end{center}

\noindentВ обоих случаях $x^{*}$ лежат в полученных интервалах неопределённости, значит можем заключить, что методы работают корректно.

\section{Дополнительные исследования}
\noindent Попытаемся ответить на вопрос: сколько чисел Фибоначчи выгоднее брать 5 или 15 и почему?\\
Так как $N$ вычислений $f(x)$ позволяют выполнить $N-1$ итерации метода Фибоначчи, то достигнутая в результате этих вычислений точность определения $x^{*}$ составляет
$$\varepsilon(N)=(\frac{\sqrt{5}-1}{2})^{N-1}\cdot (b-a)$$
$$\varepsilon(5)=(0.61803)^{5-1}\cdot (1.5-1)=0.07294901687$$
$$\varepsilon(15)=(0.61803)^{15-1}\cdot (1.5-1)=0.00059312064$$
С ростом числа $N$ точность $\varepsilon$ будет улучшаться, значит выгоднее брать 15 чисел.



\section{Программная реализация}
\noindent В процессе реализации алгоритмов использовался язык программирования Python3.6. Для проверки полученных решений пользовались пакетом MATLAB2020b.
\\\\
\noindent Исходный код находится в системе контроля версий GitHub 
\\
https://github.com/Brightest-Sunshine/Optimization-methods-2021

\begin{thebibliography}{9}
\bibitem{book1} Лесин В. В., Лисовец Ю. П. Основы методов оптимизации: Учебное пособие. 3-е изд., испр, -- СПб.: Издательство <<Лань>>, 2011. -- 352с.

\end{thebibliography}
\end{document}
